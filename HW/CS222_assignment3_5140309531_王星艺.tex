\documentclass[12pt,a4paper]{article}
\usepackage{ctex}
\usepackage{amsmath,amscd,amsbsy,amssymb,latexsym,url,bm,amsthm}
\usepackage{epsfig,graphicx,subfigure}
\usepackage{enumitem,balance,mathtools}
\usepackage{wrapfig}
\usepackage{mathrsfs, euscript}
\usepackage[usenames]{xcolor}
\usepackage{hyperref}
%\usepackage{algorithm}
%\usepackage{algorithmic}
%\usepackage[vlined,ruled,commentsnumbered,linesnumbered]{algorithm2e}
\usepackage[ruled,lined,boxed,linesnumbered]{algorithm2e}

\newtheorem{theorem}{Theorem}[section]
\newtheorem{lemma}[theorem]{Lemma}
\newtheorem{proposition}[theorem]{Proposition}
\newtheorem{corollary}[theorem]{Corollary}
\newtheorem{exercise}{Exercise}[section]
\newtheorem*{solution}{Solution}

\renewcommand{\thefootnote}{\fnsymbol{footnote}}

\newcommand{\postscript}[2]
 {\setlength{\epsfxsize}{#2\hsize}
  \centerline{\epsfbox{#1}}}

\renewcommand{\baselinestretch}{1.0}

\setlength{\oddsidemargin}{-0.365in}
\setlength{\evensidemargin}{-0.365in}
\setlength{\topmargin}{-0.3in}
\setlength{\headheight}{0in}
\setlength{\headsep}{0in}
\setlength{\textheight}{10.1in}
\setlength{\textwidth}{7in}
\makeatletter \renewenvironment{proof}[1][Proof] {\par\pushQED{\qed}\normalfont\topsep6\p@\@plus6\p@\relax\trivlist\item[\hskip\labelsep\bfseries#1\@addpunct{.}]\ignorespaces}{\popQED\endtrivlist\@endpefalse} \makeatother
\makeatletter
\renewenvironment{solution}[1][Solution] {\par\pushQED{\qed}\normalfont\topsep6\p@\@plus6\p@\relax\trivlist\item[\hskip\labelsep\bfseries#1\@addpunct{.}]\ignorespaces}{\popQED\endtrivlist\@endpefalse} \makeatother
\begin{document}
\noindent

%========================================================================
\noindent\framebox[\linewidth]{\shortstack[c]{
\Large{\textbf{CS222 Homework 3}}\vspace{1mm}\\
Name:������\ StudentID:5140309531}}
~\\

\begin{enumerate}

\item You are given coins of different denominations and a total amount of money amount.

Write a function to compute the fewest number of coins that you need to make up that amount.

If that amount of money cannot be made up by any combination of the coins, return -1.

Example 1:

coins = [1, 2, 5], amount = 11

return 3 (11 = 5 + 5 + 1)

Example 2:

coins = [2], amount = 3

return -1.

Input:

int coins[];

int n: length of coins[];

int amount;

Output:

int num;

~\\

\begin{algorithm}
    \caption{Solution 1}
    \SetKwFunction{min}{min}

    \KwIn{int coins[]; int n: length of coins[]; int amount;}
    \KwOut{int num;}
    \BlankLine

    int dp[amount + 1] $\longleftarrow$ amount + 1\;
    dp[0] $\leftarrow$ 0\;
    \For{$i\leftarrow 1$ \KwTo $amount$}{
        \For{$j\leftarrow 0$ \KwTo $n-1$}{
            \If{i $>=$ coins[j]}{
                dp[i] $\leftarrow$ \min{dp[i], dp[i - coins[j]] + 1}\;
            }
        }
    }
    \uIf{dp[amount] $>$ amount}{
        return -1\;
    }
    \Else{
        return dp[amount]\;
    }

\end{algorithm}

~\\
~\\

\item Given a string s, partition s such that every substring of the partition is a palindrome.

Return the minimum cuts needed for a palindrome partitioning of s.

For example, given s = "aab",

Return 1 since the palindrome partitioning ["aa","b"] could be produced using 1 cut.

Input:

string s;

Output:

int cuts;

\begin{algorithm}
    \caption{Solution 2}
    \SetKw{True}{True}\SetKw{False}{False}
    \SetKwFunction{min}{min}

    \KwIn{string s;}
    \KwOut{int cuts;}
    \BlankLine

    int n $\longleftarrow$ s.length()\;
    bool isPal[n][n] $\longleftarrow$ \False\;
    int cut[n]\;
    \For{$j\leftarrow 0$ \KwTo $n-1$}{
        cut[j]$\leftarrow$ j\;
        \For{$i\leftarrow 0$ \KwTo $j$}{
            \If{$s[i]==s[j] \&\& (j - i <= 1 || isPal[i + 1][j - 1])$}{
                isPal[i][j]$\leftarrow$ \True;
                \uIf{i == 0}{
                    cut[j]$\leftarrow$ 0\;
                }
                \Else{
                    cut[j]$\leftarrow$ \min{cut[j], cut[i - 1] + 1}\;
                }
            }
        }
    }
    return cut[n - 1]\;

\end{algorithm}

~\\

\item Given two arrays of length m and n with digits 0-9 representing two numbers.

Create the maximum number of length $ k \leq m + n $ from digits of the two.

The relative order of the digits from the same array must be preserved.

Return an array of the k digits.

You should try to optimize your time and space complexity.

Example 1:

nums1 = [3, 4, 6, 5]

nums2 = [9, 1, 2, 5, 8, 3]

k = 5

return [9, 8, 6, 5, 3]

Example 2:

nums1 = [6, 7]

nums2 = [6, 0, 4]

k = 5

return [6, 7, 6, 0, 4]

Example 3:

nums1 = [3, 9]

nums2 = [8, 9]

k = 3

return [9, 8, 9]

Input:

int nums1[], int m;

int nums2[], int n;

int k;

Output:

int nums[];

\begin{function}
    \caption{Greater(int *nums1, int *nums2, int i, int j)}
    \SetKw{True}{True}

    int n1$\longleftarrow$ nums1.size()\;
    int n2$\longleftarrow$ nums2.size()\;
    \While{$i < n1 \&\& j < n2 \&\& nums1[i] == nums2[j]$}{
        i$\leftarrow$ i + 1\;
        j$\leftarrow$ j + 1\;
    }
    \If{$j == n2 || (i < n1 \&\& nums1[i] > num2[j])$}{
        return \True\;
    }
\end{function}

\begin{function}
    \caption{merge(int *stack1, int *stack2)}

    int p$\longleftarrow$ 0\;
    int q$\longleftarrow$ 0\;
    int n$\longleftarrow$ stack1.size() + stack2.size()\;
    int stack[n]\;
    \For{$i\leftarrow 0$ \KwTo $n - 1$}{
        \uIf{\Greater{stack1, stack2, p, q}}{
            stack[i]$\leftarrow$ stack1[p]\;
            p$\leftarrow$ p + 1\;
        }
        \Else{
            stack[i]$\leftarrow$ stack2[q]\;
            q$\leftarrow$ q + 1\;
        }
    }
\end{function}

\begin{function}
    \caption{maxSubarray(int *nums, int k)}

    int n$\longleftarrow$ nums.size()\;
    int stack[n]\;
    int r$\longleftarrow$ 0\;
    \For{$i\leftarrow 0$ \KwTo $n - 1$}{
        \While{$n-i>k-r \&\& r>0 \&\& nums[i]>stack[r-1]$}{
            r$\leftarrow$ r - 1\;
        }
        \If{$r < k$}{
            stack[r]$\leftarrow$ nums[i]\;
            r$\leftarrow$ r + 1\;
        }
    }
    return stack\;
\end{function}

\begin{algorithm}
    \caption{Solution 3}
    \SetKwFunction{max}{max}\SetKwFunction{min}{min}

    \KwIn{int nums1[], int m; int nums2[], int n; int k;}
    \KwOut{int nums[];}
    \BlankLine

    int start$\longleftarrow$ \max{0, k - n}\;
    int end$\longleftarrow$ \min{m, k}\;
    \For{$i\leftarrow start$ \KwTo $end$}{
        int stack1[i]\;
        int stack2[k-i]\;
        int nums\_tmp[k]\;
        stack1$\leftarrow$ \maxSubarray{nums1, i}\;
        stack2$\leftarrow$ \maxSubarray{nums2, k-i}\;
        nums\_tmp$\leftarrow$ \merge{stack1, stack2}\;
        \If{\Greater{nums\_tmp, nums, 0, 0}}{
            nums$\leftarrow$ nums\_tmp\;
        }
    }
    return nums\;
\end{algorithm}

\end{enumerate}
%========================================================================
\end{document}
