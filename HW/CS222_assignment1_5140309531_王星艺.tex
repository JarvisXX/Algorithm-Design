\documentclass[12pt,a4paper]{article}
\usepackage{ctex}
\usepackage{amsmath,amscd,amsbsy,amssymb,latexsym,url,bm,amsthm}
\usepackage{epsfig,graphicx,subfigure}
\usepackage{enumitem,balance,mathtools}
\usepackage{wrapfig}
\usepackage{mathrsfs, euscript}
\usepackage[usenames]{xcolor}
\usepackage{hyperref}
%\usepackage{algorithm}
%\usepackage{algorithmic}
%\usepackage[vlined,ruled,commentsnumbered,linesnumbered]{algorithm2e}
\usepackage[ruled,lined,boxed,linesnumbered]{algorithm2e}

\newtheorem{theorem}{Theorem}[section]
\newtheorem{lemma}[theorem]{Lemma}
\newtheorem{proposition}[theorem]{Proposition}
\newtheorem{corollary}[theorem]{Corollary}
\newtheorem{exercise}{Exercise}[section]
\newtheorem*{solution}{Solution}

\renewcommand{\thefootnote}{\fnsymbol{footnote}}

\newcommand{\postscript}[2]
 {\setlength{\epsfxsize}{#2\hsize}
  \centerline{\epsfbox{#1}}}

\renewcommand{\baselinestretch}{1.0}

\setlength{\oddsidemargin}{-0.365in}
\setlength{\evensidemargin}{-0.365in}
\setlength{\topmargin}{-0.3in}
\setlength{\headheight}{0in}
\setlength{\headsep}{0in}
\setlength{\textheight}{10.1in}
\setlength{\textwidth}{7in}
\makeatletter \renewenvironment{proof}[1][Proof] {\par\pushQED{\qed}\normalfont\topsep6\p@\@plus6\p@\relax\trivlist\item[\hskip\labelsep\bfseries#1\@addpunct{.}]\ignorespaces}{\popQED\endtrivlist\@endpefalse} \makeatother
\makeatletter
\renewenvironment{solution}[1][Solution] {\par\pushQED{\qed}\normalfont\topsep6\p@\@plus6\p@\relax\trivlist\item[\hskip\labelsep\bfseries#1\@addpunct{.}]\ignorespaces}{\popQED\endtrivlist\@endpefalse} \makeatother
\begin{document}
\noindent

%========================================================================
\noindent\framebox[\linewidth]{\shortstack[c]{
\Large{\textbf{CS222 Homework 1}}\vspace{1mm}\\
Name:������\ StudentID:5140309531}}
~\\

\begin{enumerate}

\item Given an array of non-negative integers, you are initially positioned at the first index of the array.

Each element in the array represents your maximum jump length at that position.

Determine if you are able to reach the last index.

For example:

A = [2,3,1,1,4], return true.

A = [3,2,1,0,4], return false.

Input:

int A[]: the input array.

int N: length of A.

Output:

return true or false.

~\\

Answer:

\IncMargin{1em}
\begin{algorithm}
    \SetKwInOut{Input}{input}\SetKwInOut{Output}{output}

    \Input{int A[]: the input array; int N: length of A}
    \Output{return true or false}
    \BlankLine

    \emph{initialization}\:
    int cover $\longleftarrow$ 0\;
    \BlankLine
    \For{$start\leftarrow 0$ \KwTo $N-1$}{
        \If{start $>$ cover}{return false\;}
        \If{A[start]+start $>$ cover}{$maxCover \leftarrow A[start]+start$\;}
        \If{cover $>=$ N-1}{return true\;}
    }
    return false\;
\end{algorithm}

~\\
~\\

\newpage
\item Given an array of non-negative integers, you are initially positioned at the first index of the array.

Each element in the array represents your maximum jump length at that position.

Your goal is to reach the last index in the minimum number of jumps.

For example:

Given array A = [2,3,1,1,4]

The minimum number of jumps to reach the last index is 2. (Jump 1 step from index 0 to 1, then 3 steps to the last index.)

Note:

You can assume that you can always reach the last index.

Input:

int A[]: the input array.

int N: length of A.

Output:

return minimum number of jumps.

~\\

Answer:

\IncMargin{1em}
\begin{algorithm}
    \SetKwFunction{max}{max}
    \SetKwInOut{Input}{input}\SetKwInOut{Output}{output}

    \Input{int A[]: the input array; int N: length of A}
    \Output{return minimum number of jumps}
    \BlankLine

    \emph{initialization}\:
    int jump $\longleftarrow$ 0\;
    int last $\longleftarrow$ 0\;
    int current $\longleftarrow$ 0\;
    \BlankLine
    \For{$start\leftarrow 0$ \KwTo $N-1$}{
        \If{start $>$ last}{
            $last \leftarrow current$\;
            $jump \leftarrow jump + 1$\;
        }
        $current \leftarrow$ \max{current, i+A[i]}\;
    }
    return jump\;
\end{algorithm}

~\\
~\\

\newpage
\item There are N children standing in a line. Each child is assigned a rating value.

You are giving candies to these children subjected to the following requirements:

a. Each child must have at least one candy.

b. Children with a higher rating get more candies than their neighbors.(assume no equal rating neighbors)

What is the minimum candies you must give?

Input:

int A[]: the input array of rating values.

int N: length of A,(number of children).

Output:

return minimum number of candies you must give.

~\\

Answer:

\IncMargin{1em}
\begin{algorithm}
    \SetKwInOut{Input}{input}\SetKwInOut{Output}{output}

    \Input{int A[]: the input array of rating values; int N: length of A(number of children)}
    \Output{return minimum number of candies you must give}
    \BlankLine

    \emph{initialization}\:
    int candy[N] $\longleftarrow$ 0\;
    int total $\longleftarrow$ 0\;
    \BlankLine
    $candy[0] \leftarrow 1$\;
    \For{$i\leftarrow 1$ \KwTo $N-1$}{
        \lIf{A[i] $>$ A[i-1]}{$candy[i] \leftarrow candy[i-1]+1$\;}
        \lElse{$candy[i] \leftarrow 1$\;}
    }
    $total \leftarrow candy[N-1]$\;
    \For{$i\leftarrow N-2$ \KwTo 0}{
        \If{A[i] $>$ A[i+1] \&\& candy[i+1]+1 $>$ candy[i]}{$candy[i] \leftarrow candy[i+1]+1$\;}
        $total \leftarrow total + candy[i]$\;
    }
    return total\;
\end{algorithm}

~\\
~\\

\end{enumerate}
%========================================================================
\end{document}
