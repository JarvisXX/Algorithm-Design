\documentclass[12pt,a4paper]{article}
\usepackage{ctex}
\usepackage{amsmath,amscd,amsbsy,amssymb,latexsym,url,bm,amsthm}
\usepackage{epsfig,graphicx,subfigure}
\usepackage{enumitem,balance,mathtools}
\usepackage{wrapfig}
\usepackage{mathrsfs, euscript}
\usepackage[usenames]{xcolor}
\usepackage{hyperref}
%\usepackage{algorithm}
%\usepackage{algorithmic}
%\usepackage[vlined,ruled,commentsnumbered,linesnumbered]{algorithm2e}
\usepackage[ruled,lined,boxed,linesnumbered]{algorithm2e}

\newtheorem{theorem}{Theorem}[section]
\newtheorem{lemma}[theorem]{Lemma}
\newtheorem{proposition}[theorem]{Proposition}
\newtheorem{corollary}[theorem]{Corollary}
\newtheorem{exercise}{Exercise}[section]
\newtheorem*{solution}{Solution}

\renewcommand{\thefootnote}{\fnsymbol{footnote}}

\newcommand{\postscript}[2]
 {\setlength{\epsfxsize}{#2\hsize}
  \centerline{\epsfbox{#1}}}

\renewcommand{\baselinestretch}{1.0}

\setlength{\oddsidemargin}{-0.365in}
\setlength{\evensidemargin}{-0.365in}
\setlength{\topmargin}{-0.3in}
\setlength{\headheight}{0in}
\setlength{\headsep}{0in}
\setlength{\textheight}{10.1in}
\setlength{\textwidth}{7in}
\makeatletter \renewenvironment{proof}[1][Proof] {\par\pushQED{\qed}\normalfont\topsep6\p@\@plus6\p@\relax\trivlist\item[\hskip\labelsep\bfseries#1\@addpunct{.}]\ignorespaces}{\popQED\endtrivlist\@endpefalse} \makeatother
\makeatletter
\renewenvironment{solution}[1][Solution] {\par\pushQED{\qed}\normalfont\topsep6\p@\@plus6\p@\relax\trivlist\item[\hskip\labelsep\bfseries#1\@addpunct{.}]\ignorespaces}{\popQED\endtrivlist\@endpefalse} \makeatother
\begin{document}
\noindent

%========================================================================
\noindent\framebox[\linewidth]{\shortstack[c]{
\Large{\textbf{CS222 Homework 3}}\vspace{1mm}\\
Exercises for Algorithm Design and Analysis by Li Jiang, 2016 Autumn Semester}}
~\\
\begin{enumerate}

\item You are given coins of different denominations and a total amount of money amount.

Write a function to compute the fewest number of coins that you need to make up that amount.

If that amount of money cannot be made up by any combination of the coins, return -1.

Example 1:

coins = [1, 2, 5], amount = 11

return 3 (11 = 5 + 5 + 1)

Example 2:

coins = [2], amount = 3

return -1.

Input:

int coins[];

int n: length of coins[];

int amount;

Output:

int num;

%Your answer should be written here.

~\\
~\\
~\\
~\\

\item Given a string s, partition s such that every substring of the partition is a palindrome.

Return the minimum cuts needed for a palindrome partitioning of s.

For example, given s = "aab",

Return 1 since the palindrome partitioning ["aa","b"] could be produced using 1 cut.

Input:

string s;

Output:

int cuts;

%Your answer should be written here.

~\\
~\\
~\\
~\\

\item Given two arrays of length m and n with digits 0-9 representing two numbers.

Create the maximum number of length $ k \leq m + n $ from digits of the two.

The relative order of the digits from the same array must be preserved.

Return an array of the k digits.

You should try to optimize your time and space complexity.

Example 1:

nums1 = [3, 4, 6, 5]

nums2 = [9, 1, 2, 5, 8, 3]

k = 5

return [9, 8, 6, 5, 3]

Example 2:

nums1 = [6, 7]

nums2 = [6, 0, 4]

k = 5

return [6, 7, 6, 0, 4]

Example 3:

nums1 = [3, 9]

nums2 = [8, 9]

k = 3

return [9, 8, 9]

Input:

int nums1[], int m;

int nums2[], int n;

int k;

Output:

int nums[];

%Your answer should be written here.

~\\
~\\


\end{enumerate}
%========================================================================
\end{document}