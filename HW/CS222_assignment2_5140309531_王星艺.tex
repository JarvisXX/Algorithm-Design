\documentclass[12pt,a4paper]{article}
\usepackage{ctex}
\usepackage{amsmath,amscd,amsbsy,amssymb,latexsym,url,bm,amsthm}
\usepackage{epsfig,graphicx,subfigure}
\usepackage{enumitem,balance,mathtools}
\usepackage{wrapfig}
\usepackage{mathrsfs, euscript}
\usepackage[usenames]{xcolor}
\usepackage{hyperref}
%\usepackage{algorithm}
%\usepackage{algorithmic}
%\usepackage[vlined,ruled,commentsnumbered,linesnumbered]{algorithm2e}
\usepackage[ruled,lined,boxed,linesnumbered]{algorithm2e}

\newtheorem{theorem}{Theorem}[section]
\newtheorem{lemma}[theorem]{Lemma}
\newtheorem{proposition}[theorem]{Proposition}
\newtheorem{corollary}[theorem]{Corollary}
\newtheorem{exercise}{Exercise}[section]
\newtheorem*{solution}{Solution}

\renewcommand{\thefootnote}{\fnsymbol{footnote}}

\newcommand{\postscript}[2]
 {\setlength{\epsfxsize}{#2\hsize}
  \centerline{\epsfbox{#1}}}

\renewcommand{\baselinestretch}{1.0}

\setlength{\oddsidemargin}{-0.365in}
\setlength{\evensidemargin}{-0.365in}
\setlength{\topmargin}{-0.3in}
\setlength{\headheight}{0in}
\setlength{\headsep}{0in}
\setlength{\textheight}{10.1in}
\setlength{\textwidth}{7in}
\makeatletter \renewenvironment{proof}[1][Proof] {\par\pushQED{\qed}\normalfont\topsep6\p@\@plus6\p@\relax\trivlist\item[\hskip\labelsep\bfseries#1\@addpunct{.}]\ignorespaces}{\popQED\endtrivlist\@endpefalse} \makeatother
\makeatletter
\renewenvironment{solution}[1][Solution] {\par\pushQED{\qed}\normalfont\topsep6\p@\@plus6\p@\relax\trivlist\item[\hskip\labelsep\bfseries#1\@addpunct{.}]\ignorespaces}{\popQED\endtrivlist\@endpefalse} \makeatother
\begin{document}
\noindent

%========================================================================
\noindent\framebox[\linewidth]{\shortstack[c]{
\Large{\textbf{CS222 Homework 2}}\vspace{1mm}\\
Name:������\ StudentID:5140309531}}
~\\

\begin{enumerate}

\item There are two sorted arrays nums1 and nums2 of size m and n respectively.

Find the median of the two sorted arrays. The overall run time complexity should be O(log (m+n)).

Example 1:

nums1 = [1, 3]

nums2 = [2]

The median is 2.0

Example 2:

nums1 = [1, 2]

nums2 = [3, 4]

The median is (2 + 3)/2 = 2.5

Input:

int nums1[]; int m;

int nums2[]; int n;

Output:

double median.

~\\

\IncMargin{1em}
\begin{function}
    \caption{FindKth(int *nums1, int *nums2, int m, int n, int k)}
    \SetKwFunction{min}{min}

    \If{m $>$ n}{return \FindKth{nums2, nums1, n, m, k}\;}
    \If{m == 0}{return nums2[k - 1]\;}
    \If{k == 1}{return \min{nums1[0], nums2[0]}\;}
    int p $\longleftarrow$ \min{m, k / 2}\;
    int q $\longleftarrow$ k - p\;
    \uIf{nums1[p - 1] $<$ nums2[q - 1]}{return \FindKth{nums1 + p, nums2, m - p, n, k - p}\;}
    \uElseIf{nums1[p - 1] $>$ nums2[q - 1]}{return \FindKth{nums1, nums2 + q, m, n - q, k - q}\;}
    \Else{return nums1[p - 1]\;}
\end{function}

\begin{algorithm}
    \caption{Solution 1}

    \KwIn{int nums1[]; int m; int nums2[]; int n;}
    \KwOut{double median.}
    \BlankLine

    int k $\longleftarrow$ (m + n) / 2\;
    \uIf{(m + n) \% 2 == 0}{
        return (\FindKth{nums1, nums2, m, n, k} + \FindKth{nums1, nums2, m, n, k + 1}) / 2\;
    }
    \Else{return \FindKth{nums1, nums2, m, n, k+1}\;}
\end{algorithm}

~\\
~\\


\item Find the contiguous subarray within an array (containing at least one number) which has the largest sum.

For example, given the array [-2,1,-3,4,-1,2,1,-5,4], the contiguous subarray [4,-1,2,1] has the largest sum = 6.

Input:

int A[]: the input array.

int N: length of A.

Output:

return the largest sum.

~\\

\begin{algorithm}
    \caption{Solution 2}
    \SetKwFunction{max}{max}

    \KwIn{int A[]: the input array; int N: length of A}
    \KwOut{return the largest sum}
    \BlankLine
    
    int tmp $\longleftarrow$ A[0]\;
    int result $\longleftarrow$ tmp\;
    \For{$i\leftarrow 1$ \KwTo $N-1$}{
        \uIf{tmp $>$ 0}{tmp $\leftarrow$ tmp + A[i]\;}
        \Else{tmp $\leftarrow$ A[i]\;}
        result $\leftarrow$ \max{result, tmp}\;
    }
    return result\;
\end{algorithm}

~\\
~\\


\item Given a non-empty array containing only positive integers, find if the array can be partitioned into two subsets such that the sum of elements in both subsets is equal.

Note:

Each of the array element will not exceed 100.

The array size will not exceed 200.

Example 1:

Input: [1, 5, 11, 5]

Output: true

Explanation: The array can be partitioned as [1, 5, 5] and [11].

Example 2:

Input: [1, 2, 3, 5]

Output: false

Explanation: The array cannot be partitioned into equal sum subsets.

Input:

int A[]: the input array.

int N: length of A.

Output:

return true or false.

\begin{algorithm}
    \caption{Solution 3}

    \KwIn{int A[]: the input array; int N: length of A}
    \KwOut{return true or false}
    \BlankLine
    
    \If{the sum of A is odd}{return false\;}
    int dp[sum / 2 + 1] $\longleftarrow$ 0\;
    dp[0] $\leftarrow$ 1\;
    \For{$i\leftarrow 0$ \KwTo $N-1$}{
        \For{$j\leftarrow sum / 2$ \KwTo $A[i]$}{
            dp[j] $\leftarrow$ dp[j] $||$ dp[j-A[i]]\;
        }
    }
    \uIf{dp[sum / 2] == 1}{return true\;}
    \Else{return false\;}
\end{algorithm}

~\\
~\\

\end{enumerate}
%========================================================================
\end{document}
